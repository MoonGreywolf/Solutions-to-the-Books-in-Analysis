\documentclass[12pt, a4paper, oneside]{ctexart}
\usepackage{amsmath, amsthm, amssymb, bm, color, framed, graphicx, hyperref, mathrsfs}

\title{\textbf{Fourier Analysis Chapter 1}}
\author{张浩然}
\date{\today}
\linespread{1.5}
\definecolor{shadecolor}{RGB}{241, 241, 255}
\newcounter{problemname}
\newenvironment{problem}{\begin{shaded}\stepcounter{problemname}\par\noindent\textbf{题目\arabic{problemname}. }}{\end{shaded}\par}
\newenvironment{solution}{\par\noindent\textbf{解答. }}{\par}
\newenvironment{note}{\par\noindent\textbf{题目\arabic{problemname}的注记. }}{\par}

\begin{document}

\maketitle



\begin{problem}
Exercise 6:
\par
若$f$在$\mathbb{R}$上二阶连续可导,且为微分方程
$$
f''(t)+c^2f(t)=0
$$
\par
的一个解,$c\neq 0$,证明:
存在$a,b\in \mathbb{R}$使得,$f(t)=a\cos ct+b\sin ct$
\end{problem}


\begin{solution}
\par
学过ODE的基本理论的都是知道这是一个二阶线性齐次常系数微分方程,
基础解系很明显,不过我们这里给一个构造性的直接法.
\par
令$$
\begin{aligned}
  \begin{cases}
  &g(t)=f(t)\cos ct -c^{-1}f'(t)\sin ct\\
  &h(t)= f(t)\sin ct +c^{-1}f'(t)\cos ct
  \end{cases}
\end{aligned}
$$
\par
对t求导,
$$
\begin{aligned}
  \begin{cases}
  &g'(t)=-cf(t)\sin ct -c^{-1}f''(t)\sin ct =cf(t)\sin ct-cf(t)\sin ct=0\\
  &h'(t)= cf(t)\cos ct +c^{-1}f''(t)\cos ct= cf(t)\cos ct - cf(t)\cos ct =0
  \end{cases}
\end{aligned}
$$
于是两个函数为常数$a,b$
$$
\begin{aligned}
  \begin{cases}
  &f(t)\cos ct -c^{-1}f'(t)\sin ct=a\\
  &f(t)\sin ct +c^{-1}f'(t)\cos ct=b
  \end{cases}
\end{aligned}
$$
\par
利用Cramer法则,得到$f(t)=a\cos t+b\sin t $.
\end{solution}

\begin{problem}
Exercise 9:
\par
对于拨弦问题,即
$$
\begin{cases}
  u_{tt}=u_{xx}\\
  u(x,0)=f(x)\\
  u_{t}(x,0)=0
\end{cases}
$$
\par
其中,如果初始形状$f(x)$为
$$
f(x)=
\begin{cases}
  \dfrac{xh}{p},  &0\leqslant x \leqslant p\\
  \dfrac{h(\pi -x)}{\pi -p},& p\leqslant x \leqslant \pi
\end{cases}
$$
\par
证明:$f$的正弦Fourier系数是$A_m=\dfrac{2h}{m^2}\dfrac{\sin mp}{p(\pi-p)}$.
\par
并找出第$2,4,…,2n、…$谐波消失的位置以及第$3,6,…,3n,…$谐波消失的位置.
\end{problem}


\begin{solution}

$$
\begin{aligned}
A_m &=\dfrac{2}{\pi}\int_{0}^{\pi}f(x)\sin mx \mathrm{d}x\\
    &=\dfrac{2}{\pi}\int_{0}^{p}\dfrac{xh}{p}\sin mx \mathrm{d}x
    +\dfrac{2}{\pi}\int_{p}^{\pi}\dfrac{h(\pi -x)}{\pi -p}\sin mx \mathrm{d}x\\
    &=\dfrac{2h}{\pi p}\left(-p\dfrac{\cos mp}{m}+\dfrac{\sin mp}{m^2} \right)
    +\dfrac{2h}{\pi (\pi -p)}\left((\pi -p)\dfrac{\cos mp}{n}+\dfrac{\sin mp}{m^2}\right)\\
    &=\dfrac{2h}{m^2}\dfrac{\sin mp}{p(\pi-p)}.
\end{aligned}
$$

\par
对于$2n$,谐波消失位置只可能为满足$\sin 2np=0$,即$p=\dfrac{k\pi}{2n},
1\leqslant k \leqslant 2n-1$.
\par
$\bigcap_{n=1}^{\infty}\left\{\dfrac{k\pi}{2n}\right\}=\left\{\dfrac{\pi}{2}\right\}$

\par
对于$3n$,谐波消失位置只可能为满足$\sin 3np=0$,即$p=\dfrac{k\pi}{3n},
1\leqslant k \leqslant 3n-1$.
\par
$\bigcap_{n=1}^{\infty}\left\{\dfrac{k\pi}{3n}\right\}
=\left\{\dfrac{\pi}{3}\right\}\cup\left\{\dfrac{2\pi}{3}\right\}$

\end{solution}


\begin{problem}
  Exercise 10:
  \par
给出Laplace算子:$$
\Delta =\dfrac{\partial^2}{\partial x^2}+\dfrac{\partial^2}{\partial y^2}
$$
\par
的极坐标表达式.
\par
并证明:$$
\left|\dfrac{\partial u}{\partial x}\right|^2+\left|\dfrac{\partial u}{\partial y}\right|^2
=\left|\dfrac{\partial u}{\partial r}\right|^2+\dfrac{1}{r^2}\left|\dfrac{\partial u}{\partial \theta}\right|^2
$$

\end{problem}


\begin{solution}
  \par
  这是一道简单的《数学分析三》练习题
  \par
  $r=\sqrt{x^2+y^2}, \theta =\arctan\left(\dfrac{y}{x}\right)$
  \par
  $$\dfrac{\partial r}{\partial x}=\cos \theta,
   \dfrac{\partial \theta}{\partial x}=-\dfrac{\sin \theta}{r}$$
   \par
    $$\dfrac{\partial r}{\partial y}=\sin \theta,
   \dfrac{\partial \theta}{\partial y}=\dfrac{\cos \theta}{r}$$
  \par
  复合求导公式:
  $$
   \begin{aligned}
   &\dfrac{\partial}{\partial x}=\cos \theta \dfrac{\partial}{\partial r}
   -\dfrac{\sin \theta}{r}\dfrac{\partial}{\partial \theta}\\
   &\dfrac{\partial}{\partial y}=\sin \theta \dfrac{\partial}{\partial r}
   +\dfrac{\cos \theta}{r}\dfrac{\partial}{\partial \theta}\\
   \end{aligned}
  $$
\par
于是,
$$
\left|\dfrac{\partial u}{\partial x}\right|^2+\left|\dfrac{\partial u}{\partial y}\right|^2
=\left|\dfrac{\partial u}{\partial r}\right|^2+\dfrac{1}{r^2}\left|\dfrac{\partial u}{\partial \theta}\right|^2
$$
\par
进一步,
$$
   \begin{aligned}
   &\dfrac{\partial^2}{\partial x^2}=\cos^2 \theta \dfrac{\partial^2}{\partial r^2}
   +\dfrac{\sin^2 \theta}{r^2}\dfrac{\partial^2}{\partial \theta^2}+
   \dfrac{2\sin \theta \cos \theta}{r^2}\dfrac{\partial}{\partial \theta}
   +\dfrac{\sin^2 \theta}{r}\dfrac{\partial}{\partial r}
   -\dfrac{2\sin \theta \cos \theta}{r^2}\dfrac{\partial^2}{\partial r\partial \theta}\\
   &\dfrac{\partial^2}{\partial y^2}=\sin^2 \theta \dfrac{\partial^2}{\partial r^2}
   +\dfrac{\cos^2 \theta}{r^2}\dfrac{\partial^2}{\partial \theta^2}-
   \dfrac{2\sin \theta \cos \theta}{r^2}\dfrac{\partial}{\partial \theta}
   +\dfrac{\cos^2 \theta}{r}\dfrac{\partial}{\partial r}
   +\dfrac{2\sin \theta \cos \theta}{r^2}\dfrac{\partial^2}{\partial r\partial \theta}
   \end{aligned}
  $$
  \par
  加起来就是,
  $$
\dfrac{\partial^2}{\partial x^2}+\dfrac{\partial^2}{\partial y^2}
=\dfrac{\partial^2}{\partial r^2}+\dfrac{1}{r}\dfrac{\partial}{\partial r}+\dfrac{1}{r^2}\dfrac{\partial^2}{\partial \theta^2}
  $$
\end{solution}
\begin{note}
   \par
   我们用分离变量的方法解热稳态方程就是直角坐标Laplace算子变成极坐标下Laplace算子,
   最后解一个二阶常系数线性齐次微分方程和一个二阶欧拉方程.

\end{note}



\begin{problem}
Exercise 11:
\par
证明当$n \in \mathbb{Z}$时, 二阶微分方程:
$$
r^2F''(r)+rF'(r)-n^2F(r)=0
$$
\par
的解必定为$r^n$和$r^{-n}$的线性组合$(n\neq 0)$.
\par
或$1$和$\log r$的线性组合$(n=0)$
\end{problem}

\begin{solution}
\par
教材上给了一种降阶的方法,最后归结为一阶非齐次变系数微分方程,
这里直接用解欧拉方程的方法迅速给出解系.
\par
令$r=\mathrm{e}^t, t=\log r$
\par
$$r^2\dfrac{\mathrm{d}^2F}{\mathrm{d}r^2}+r\dfrac{\mathrm{d}F}{\mathrm{d} r}-n^2F=0$$
\par
有:
$$
\begin{aligned}
&\dfrac{\mathrm{d}F}{\mathrm{d}r}=\dfrac{1}{r}\dfrac{\mathrm{d}F}{\mathrm{d}t}\\
&\dfrac{\mathrm{d}^2F}{\mathrm{d}r^2}=-\dfrac{1}{r^2}\dfrac{\mathrm{d}F}{\mathrm{d}t}
+\dfrac{1}{r^2}\dfrac{\mathrm{d}^2F}{\mathrm{d}t^2}\\
\end{aligned}
$$
\par
最后原方程变成二阶常系数齐次微分方程:
$$
\dfrac{\mathrm{d}^2F}{\mathrm{d}t^2}-n^2F=0
$$
\par
$n=0$, 有基础解系$1, t$, 即$1, \log r$.
\par
$n\neq 0$, 有基础解系$\mathrm{e}^{-n}, \mathrm{e}^{n}$即$r^n, r^{-n}$。
\end{solution}



\begin{problem} 
  Problem  1
\par
考虑Dirichlet问题,
在矩形区域$R=\{(x,y): 0\leqslant x \leqslant \pi, 0 \leqslant y \leqslant 1\}$
中有
$$
\begin{aligned}
\begin{cases}
  \Delta u=0\\
  u(x,0)=f_0(x)\\
  u(x,1)=f_1(x)\\
  u(0,y)=0\\
  u(\pi,y)=0\\
\end{cases}
\end{aligned}
$$
\par
其中$f_0,f_1$是确定解的初值.
\par
如果有Fourier展开为$f_0=\sum_{k=1}^{\infty}A_k\sin kx$,$f_1=\sum_{k=1}^{\infty}B_k\sin kx$,
利用分离变量法证明:
$$
u(x,y)=\sum_{k=1}^{\infty}\left(\dfrac{\sinh k(1-y)}{\sinh k}A_k+\dfrac{\sinh ky}{\sinh k}B_k\right)\sin kx
$$

\end{problem}

\begin{solution}
\par
分离变量法,就设$u(x,y)=F(x)G(y)$,
代入$$
\Delta u=0
$$
即$$
\dfrac{F''(x)}{F(x)}=-\dfrac{G''(y)}{G(y)}=\lambda
$$
\par
这是因为左右变量独立,求导可以看出会等于常数$\lambda$.
则
$$
\begin{aligned}
  \begin{cases}
    F''(x)-\lambda F(x)=0\\
    G''(y)+\lambda G(y)=0
  \end{cases}
\end{aligned}
$$
\par
接下来讨论我们需要的解的$\lambda$范围,我们不考虑任何平凡解:
\par
\quad
\par
(1)$\lambda =0 $
\par
解得$F(x)=C_1+C_2x$.
\par
由$u(0,y)=0, u(\pi,y)=0$,得到:$F(0)G(y)=0, F(\pi)G(y)=0$,
由于不考虑平凡解,$F(0)=0, F(\pi)=0$.
\par
$C_1=C_2=0$,进而$u(x,y)=0$,不考虑.
\par
\quad
\par
(2)$\lambda>0$
\par
解得$F(x)=C_1\mathrm{e}^{\sqrt{\lambda}x}+C_2\mathrm{e}^{-\sqrt{\lambda}x}$
\par
由$u(0,y)=0, u(\pi,y)=0$,得到:$F(0)G(y)=0, F(\pi)G(y)=0$,
由于不考虑平凡解,$F(0)=0, F(\pi)=0$.
\par
同样的方法,$C_1=C_2=0$,进而$u(x,y)=0$,不考虑.
\par
\quad
\par
(3)$\lambda<0$
\par
设$\lambda=-k^2, k\in \mathbb{N}_{\geqslant 1}$.
\par
解得:$$
\begin{aligned}
&F(x)=C_{1,k}\cos kx +C_{2,k} \sin kx\\
&G(y)=C_{3,k}\mathrm{e}^{ky}+C_{4,k}\mathrm{e}^{-ky}
\end{aligned}
$$
\par
又有:
$u(0,y)=0, u(\pi,y)=0$,
可以解出$C_{1,k}=0, \forall k \in \mathbb{N}_{\geqslant 1}$.
\par
$$
\begin{aligned}
u(x,y)&=\sum_{k=1}^{\infty}a_k\left(C_{3,k}\mathrm{e}^{ky}+C_{4,k}\mathrm{e}^{-ky}\right)C_{2,k}\sin kx\\
&=\sum_{k=1}^{\infty}\left(\mu_{1,k}\mathrm{e}^{ky}+\mu_{2,k}\mathrm{e}^{-ky}\right)\sin kx\\
\end{aligned}
$$
\par
根据$$u(x,0)=f_0(x), \quad
u(x,1)=f_1(x)$$
\par
$$
\begin{aligned}
\begin{cases}
\mu_{1,k}+\mu_{2,k}=A_k\\
\mu_{1,k}\mathrm{e}^k+\mu_{2,k}\mathrm{e}^{-k}=B_k
\end{cases}
\end{aligned}
$$
\par
解得:
$$
\begin{aligned}
  \begin{cases}
  \mu_{1,k}=\dfrac{A_k\mathrm{e}^k-B_k}{\mathrm{e}^k-\mathrm{e}^{-k}}\\
  \mu_{2,k}=\dfrac{-A_k\mathrm{e}^{-k}+B_k}{\mathrm{e}^k-\mathrm{e}^{-k}}\\
  \end{cases}
  \end{aligned}
$$
\par
于是,
$$
\begin{aligned}
u(x,y)&=\sum_{k=1}^{\infty}\left(\mu_{1,k}\mathrm{e}^{ky}+\mu_{2,k}\mathrm{e}^{-ky}\right)\sin kx\\
&=\sum_{k=1}^{\infty}\left(\dfrac{A_k\dfrac{\mathrm{e}^{k(1-y)}-\mathrm{e}^{-k(1-y)}}{2}
+B_k\dfrac{\mathrm{e}^{ky}-\mathrm{e}^{-ky}}{2}}{\dfrac{\mathrm{e}^k-\mathrm{e^{-k}}}{2}}\right)\sin kx\\
&=\sum_{k=1}^{\infty}\left(\dfrac{\sinh k(1-y)}{\sinh k}A_k+\dfrac{\sinh ky}{\sinh k}B_k\right)\sin kx
\end{aligned}
$$
\end{solution}





\end{document}