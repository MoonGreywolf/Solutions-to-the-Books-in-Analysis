\documentclass[12pt, a4paper, oneside]{ctexart}
\usepackage{amsmath, amsthm, amssymb, bm, color, framed, graphicx, hyperref, mathrsfs}

\title{\textbf{Fourier Analysis Chapter 2}}
\author{张浩然}
\date{\today}
\linespread{1.5}
\definecolor{shadecolor}{RGB}{241, 241, 255}
\newcounter{problemname}
\newenvironment{problem}{\begin{shaded}\stepcounter{problemname}\par\noindent\textbf{题目\arabic{problemname}. }}{\end{shaded}\par}
\newenvironment{solution}{\par\noindent\textbf{解答. }}{\par}
\newenvironment{note}{\par\noindent\textbf{题目\arabic{problemname}的注记. }}{\par}

\begin{document}

\maketitle


\begin{problem}
Exercise 3:
\par
我们重返第一章的“拨动的弦”问题,证明:
初始函数$f$等于它的Fourier正弦级数即:
$$
f(x)=\sum_{m=1}^{\infty}A_m\sin mx,  A_m=\dfrac{2h}{m^2}\dfrac{\sin mp}{p(\pi-p)}
$$
\par
其中:
$$
f(x)=
\begin{cases}
    \dfrac{xh}{p} &,0\leqslant x \leqslant p;\\
    \dfrac{h(\pi-x)}{\pi-p} &,p \leqslant x \leqslant \pi.
\end{cases}
$$
\end{problem}

\begin{solution}
    证明:
    \par
    存在常数$C>0$,使得$|A_m|\leqslant \dfrac{C}{m^2}$,对任意$m \in \mathbb{N}$成立.
    \par
    又$
    \sum\frac{C}{m^2}$收敛,于是$\sum A_m $绝对收敛,即$f$的Fourier系数级数$\sum \hat{f}(n)$绝对收敛,
    又容易验证$f$在$[0,\pi]$上连续,则$f$的Fourier级数绝对收敛,并一致收敛到$f$,即
    $$
    f(x)=\sum_{m=1}^{\infty}A_m \sin mx
    $$


\end{solution}

\begin{problem}
Exercise 6:
\par
$f(\theta)=|\theta|$定义在$[-\pi,\pi]$上,则
\par
(1)画出$f(\theta)$的图像;
\par
(2)计算$\hat{f}(n)$;
\par
(3)改写Fourier级数为$\cos nx$和$\sin nx$的组合;
\par
(4)取$\theta=0$,计算:
$$
\sum_{n \geq 1, n \text{ odd}}\dfrac{1}{n^2}=\dfrac{\pi^2}{8},\quad
\sum_{n \geq 1}\dfrac{1}{n^2}=\dfrac{\pi^2}{6}
$$
\end{problem}

\begin{solution}
\par
(1)(2)我们舍去平凡计算,有
$$
\hat{f}(n)=
\begin{cases}
    \dfrac{\pi}{2} &,n=0;\\
    \dfrac{-1+(-1)^n}{\pi n^2} &,n\neq 0.
\end{cases}
$$
\par
(3)改写为:
$$
\sum_{n=-\infty}^{+\infty}\hat{f}(n)\mathrm{e}^{inx}=
\dfrac{\pi}{2}-\dfrac{4}{\pi}\sum_{k=1}^{+\infty}\dfrac{\cos (2k+1)x}{(2k+1)^2}
$$
\par
(4)既然是数学专业学生,就不应该乱代入一些值,凭什么可以计算?
\par
这是因为对任意$k\geqslant 1$有:
$$
\left|\dfrac{\cos(2k+1)x}{(2k+1)^2}\right|\leqslant \dfrac{1}{(2k+1)^2}
$$
\par
$\sum \frac{1}{(2k+1)^2}$收敛,则Fourier级数绝对收敛且一致收敛,
又因为$f(\theta)$是连续函数,则Fourier级数一致收敛到$f$,即
$$
f(\theta)=\dfrac{\pi}{2}-\dfrac{4}{\pi}\sum_{k=1}^{+\infty}\dfrac{\cos (2k+1)x}{(2k+1)^2}
$$
\par
代入$\theta =0$,有:
$$
0=\dfrac{\pi}{2}-\dfrac{4}{\pi}\sum_{k=1}^{+\infty}\dfrac{1}{(2k+1)^2}
$$
\par
即$$
\sum_{n \geqslant 1 , n \text{ odd}}\dfrac{1}{n^2}=\dfrac{\pi^2}{8}
$$
\par
又$$
\sum_{n \geqslant 1 }\dfrac{1}{n^2}=
\sum_{n \geqslant 1 , n \text{ odd}}\dfrac{1}{n^2}+
\sum_{n \geqslant 1 , n \text{ even}}\dfrac{1}{n^2}
$$
\par
这是因为右边两个级数都收敛,且对第二个级数观察发现为左边的$\frac{1}{4}$倍,
\par
$$
\dfrac{3}{4}\sum_{n \geqslant 1}\dfrac{1}{n^2}=\dfrac{\pi^2}{8}
$$
\par
则
$$
\sum_{n \geqslant 1}\dfrac{1}{n^2}=\dfrac{\pi^2}{6}
$$

\end{solution}




\begin{problem}
Exercise 8:
\par
验证: 
\par
$\dfrac{1}{2i}\sum_{n\neq 0}\dfrac{\mathrm{e}^{inx}}{n}$是周期为$2\pi$
的锯齿函数的Fourier级数,其中$f(0)=0$, $f(x)$定义为:
$$
f(x)=
\begin{cases}
-\dfrac{\pi}{2}-\dfrac{x}{2} &, -\pi <x<0;\\
 \dfrac{\pi}{2}-\dfrac{x}{2}  &, 0<x<\pi.
\end{cases}
$$
\par
显然,$f$并不连续.进一步证明:
\par
$\dfrac{1}{2i}\sum_{n\neq 0}\dfrac{\mathrm{e}^{inx}}{n}$,
对所有$x$都收敛,并且此级数在$x=0$处的值等于$\frac{f(0^+)+f(0^-)}{2}$


\end{problem}

\begin{solution}
\par
先验证之,$n=0$时,$\hat{f}(0)=0$.
$n\neq 0$时,
$$
\begin{aligned}
\hat{f}(n)
&=\dfrac{1}{2\pi}\int_{-\pi}^{\pi}f(x)\mathrm{e}^{-inx}\mathrm{d}x\\
&=\dfrac{1}{2\pi}\int_{-\pi}^{\pi}f(x)(-i\sin nx)\mathrm{d}x\\
&=\dfrac{1}{2\pi}\int_{0}^{\pi}2\left(\dfrac{\pi}{2}-\dfrac{x}{2}\right)(-i\sin nx)\mathrm{d}x\\
&=-\dfrac{i}{2\pi}\int_{0}^{\pi}(\pi \sin nx -x \sin nx)\mathrm{d}x\\
&=\dfrac{1}{2n i}
\end{aligned}
$$
\par
验证完成,下证收敛性:$x\neq 0$时,对任意$N\in \mathbb{N}^*$有:
$$
\begin{aligned}
\left|\dfrac{1}{2i}\sum_{n=-N, n \neq 0}^{N}\mathrm{e}^{inx}\right|
&\leqslant \dfrac{1}{2} \left|\mathrm{e}^{-iNx}\dfrac{1-\mathrm{e}^{iNx}}{1-\mathrm{e}^{ix}}\right|
+\dfrac{1}{2}\left |\mathrm{e}^{ix}\dfrac{1-\mathrm{e}^{iNx}}{1-\mathrm{e}^{ix}}\right|\\
&\leqslant \dfrac{2}{|1-\mathrm{e}^{ix}|}
\end{aligned}
$$
\par
可得部分和在$x\neq 0$关于$N$一致有界,并且实数列$\left\{\dfrac{1}{n}\right\}$单调收敛到$0$.
\par
\par
于是,$\dfrac{1}{2i}\sum_{n\neq 0}\dfrac{\mathrm{e}^{inx}}{n}$在$x\neq 0$处收敛.
\par
当$x=0$, $S_Nf=0$,进而$\dfrac{1}{2i}\sum_{n\neq 0}\dfrac{\mathrm{e}^{inx}}{n}$在$x=0$
处为$0=\frac{f(0^+)+f(0^-)}{2}$.
\end{solution}
\par
\quad
\par
\begin{note}
    \par
    此处改造了一下Dirichlet判别法的复数版本,对双边数列,只需要
    分而治之,化两个两个即可,并且对单调性要求也不必单减,只要在需要的范围内单调即可.
\end{note}
\par
\quad\\
\quad
\par
\begin{problem}
Exercise 9:
\par
令$f(x)=\chi_{[a,b]}(x)$为区间$[a,b]\subset [-\pi,\pi]$的特征函数,即:
$$
\chi_{[a,b]}(x)=
\begin{cases}
    1, x \in [a,b];\\
    0, x \notin [a,b].
\end{cases}
$$
\par
(1)证明:$f$的Fourier级数是:
$$
f(x)\sim
 \dfrac{b-a}{2\pi}+\sum_{n\neq 0} \dfrac{\mathrm{e}^{-ina}-\mathrm{e}^{-inb}}{2\pi in}\mathrm{e}^{inx}
$$
\par
(2)证明:若$a\neq -\pi$或$b \neq \pi$,并且$a\neq b$,
则(1)中的Fourier级数对任何$x$都不绝对收敛.
\par
(3)然而,再证明:此Fourier级数对任何$x$都收敛,并说明$a=-\pi$且$b=\pi$时发生了什么.
\end{problem}

\begin{solution}
\par
(1)
$n=0$时,
$$
\hat{f}(0)=\dfrac{1}{2\pi}\int_{-\pi}^{\pi}\chi_{[a,b]}(x)\mathrm{d}x=\dfrac{b-a}{2\pi}
$$
\par
$n\neq 0$时,
$$
\begin{aligned}
\hat{f}(n)&=\dfrac{1}{2\pi}\int_{-\pi}^{\pi}\chi_{[a,b]}(x)\mathrm{e}^{-inx}\mathrm{d}x\\
&=\dfrac{1}{2\pi}\int_{a}^{b}\mathrm{e}^{-inx}\mathrm{d}x\\
&=\dfrac{\mathrm{e}^{-ina}-\mathrm{e}^{-inb}}{2\pi i n}
\end{aligned}
$$
\par
于是确实有:
$$
f(x)\sim \dfrac{b-a}{2\pi}+\sum_{n \neq 0}\dfrac{\mathrm{e}^{-ina}-\mathrm{e}^{-inb}}{2\pi i n}\mathrm{e}^{inx}
$$
\par
(2)
\par
第二题的提示是很典型的错误,从调和级数$\sum\frac{1}{n}$中选无数项并不一定发散,比如
$\sum \frac{1}{n^2}$收敛.
\par
所以此题不能按提示那样做,我下面展示一种做法.
\par
我们不关心常数项,则:
$$
\begin{aligned}
&\sum_{n\neq 0}\left|\dfrac{\mathrm{e}^{-ina}-\mathrm{e}^{-inb}}{2\pi in}\mathrm{e}^{inx}
\right|\\
=&\sum_{n\neq 0}\left|\dfrac{\mathrm{e}^{in\frac{(b-a)}{2}}
-\mathrm{e}^{-in\frac{(b-a)}{2}}}{2\pi in}\mathrm{e}^{-\frac{a+b}{2}}\mathrm{e}^{inx}\right|\\
=&\sum_{n=1}^{\infty}\dfrac{2}{\pi}\dfrac{|\sin (\theta)|}{n}
\end{aligned}
$$
\par
其中$\theta=\dfrac{b-a}{2}\neq 0$,只用研究:
$$
\sum_{n=1}^{\infty}\dfrac{|\sin (\theta)|}{n}\geqslant \sum_{n=1}^{\infty}\dfrac{\sin^2(\theta)}{n}=
\sum_{n=1}^{\infty}\dfrac{1}{2n}-\sum_{n=1}^{\infty}\dfrac{\cos 2\theta}{2n}
$$
\par
根据经典方法,最右边第一个级数发散,第二个收敛,因为Dirichlet判别法,于是最左边级数发散,此题得证.
\par
还有一种估计方法,利用柯西收敛准则,我就不写了,因为太长了.
\par
(3)
\par
根据推广的Dirichlet判别法和特殊点单独处理,Fourier级数处处收敛.是容易仿照Exercise 8证明的.
\par
当$a=-\pi,b=\pi$时,其Fourier级数为$1$.


\end{solution}


\begin{problem}
Exercise 10:
    \par
    设$f$是周期为$2\pi$的函数且$f\in C^k$, 证明:
    $$
\hat{f}(n)=O\left(\dfrac{1}{|n|^k}\right), n \to \infty
    $$
\end{problem}

\begin{solution}
\par
熟知:
$$
\widehat{f^{(k)}}(n)=(in)^k\hat{f}(n)
$$
\par
而$f\in C^k$,则存在$C>0$,使得对任意$n\in \mathbb{N}$,有$|\widehat{f^{(k)}}(n)|\leqslant C$
$$
|\hat{f}(n)|\leqslant \dfrac{C}{|n|^k}
$$
\par
即$$
\hat{f}(n)=O\left(\dfrac{1}{|n|^k}\right)
$$
\end{solution}


\begin{problem}
Exercise 11:
\par
设$\{f_k\}_{k=1}^{\infty}$是一个在$[0,1]$上黎曼可积的函数列,
使得$$
\int_{0}^{1}|f_k(x)-f(x)|\mathrm{d}x\to 0, k \to \infty
$$
 \par
 证明:
 $$
 \hat{f_k}(n)\to \hat{f}(n), n \to \infty
 $$
 \par
 并且关于$n$是一致的.
\end{problem}

\begin{solution}
\par
我们直接做差估计:
$$
\begin{aligned}
&|\hat{f_k}(n)-\hat{f}(n)|\\
=&\left|\int_{0}^{1}f_k(x)\mathrm{e}^{-i2\pi nx}\mathrm{d}x-\int_{0}^{1}f_k(x)\mathrm{e}^{-i2\pi nx}\mathrm{d}x\right|\\
=&\int_{0}^{1}\left|f_k(x)-f(x)\right|\to 0, k\to \infty
\end{aligned}
$$
\par
并且此极限至多依赖$x$,关于$n$是一致的.
\end{solution}


\begin{problem}
Exercise 12:
\par
证明:
\par
如果复级数$\sum c_n$收敛到$s$,则$\sum c_n$的$Ces\grave{a}ro $和收敛到$s$.
    
\end{problem}

\begin{solution}
\par
这是我们《数学分析》中常见的Cauchy命题的复版本,证明方法一模一样,
所以有些人要强调《数学分析》的重要性,其实不如用比较高级的教材一次扫尽.
\par
设$S_n=\sum_{i=1}^{n}c_n$.
\par
对任意$\varepsilon>0$,存在$N_1\in \mathbb{N}$,使得对任意$n>N_1$,
$$
|S_n-s|<\dfrac{\varepsilon}{2}
$$
\par
$$
\begin{aligned}
&\left|\dfrac{S_1+S_2+\cdots+S_n}{n}-s\right|\\
=&\left|\dfrac{(S_1-s)+(S_2-s)+\cdots+(S_n-s)}{n}\right|\\
\leqslant&\dfrac{|S_1-s|+|S_2-s|+\cdots+|S_{N_1}-s|}{n}+\dfrac{n-N_1}{n}\dfrac{\varepsilon}{2}\\
<&\varepsilon
\end{aligned}
$$
\par
取$N_2 \in \mathbb{N}$使得$$
\dfrac{|S_1-s|+|S_2-s|+\cdots+|S_{N_1}-s|}{n}<\dfrac{\varepsilon}{2}
$$
\par
对上述的$\varepsilon>0$,存在$N=\max\{N_1,N_2\}$,使得$n>N$,时
$$
\left|\dfrac{S_1+S_2+\cdots+S_n}{n}-s\right|<\varepsilon
$$
\par
于是命题得证.
\end{solution}


\begin{problem}
Exercise 13:
\par
接下来,我们打算证明:$Ces\grave{a}ro$可和的条件强于$Abel$可和的条件.
\par
(1)证明:
若复级数$\sum c_n$收敛到$s$,其$Abel$和收敛到$s$.
\par
(2)证明:
存在复级数$Abel$可和,但是不收敛.
\par
(3)
证明:
若复级数$Ces\grave{a}ro$可和收敛到$\sigma$,其$Abel$和收敛到$\sigma$
\par
(4)
证明:
存在复级数$Abel$可以和,但是不$Ces\grave{a}ro$可和.

\end{problem}

\begin{solution}
(1)
\par
证明:利用$Abel$求和公式:记$s_n=\sum_{i=1}^{n}c_i$
$$
\begin{aligned}
\sum_{n=1}^{N}c_n r^n&=r^Ns_N+\sum_{n=1}^{N-1}(r^n-r^{n+1})s_n\\
&=r^{N+1}s_N+(1-r)\sum_{n=1}^{N}s_nr^n
\end{aligned}
$$
\par
由于$$
\lim_{n \to \infty}s_n=s
$$
\par
任取$\varepsilon>0$,存在$N_1\in \mathbb{N}$,使得$n>N_1$时,$|s_n-s|<\varepsilon$.
\par
存在$M>0$,使得对任意$n\in \mathbb{N}$,有$|s_n|\leqslant M, |s|\leqslant M$.
\par
于是,$r^{N+1}s_N$和$(1-r)\sum_{n=1}^{N}s_nr^n$的极限都存在.
\par
令$N\to \infty$,则
$$
\sum_{n=1}^{\infty}c_n r^n=(1-r)\sum_{n=1}^{\infty}s_nr^n
$$
\par
下面只需要证明:
$$
(1-r)\sum_{n=1}^{\infty}s_nr^n=s, r \to 1^-
$$
\par
我们直接做差估计:
$$
\begin{aligned}
&\left|(1-r)\sum_{n=1}^{\infty}s_nr^n-s\right|\\
=&\left|(1-r)\sum_{n=1}^{\infty}(s_n-s)r^n+s(1-r)\sum_{n=1}^{\infty}r^n-s\right|\\
=&\left|(1-r)\sum_{n=1}^{\infty}(s_n-s)r^n+(r-1)s\right|\\
\leqslant&\left|(1-r)\sum_{n=1}^{N_1}(s_n-s)r^n\right|
+\left|(1-r)\sum_{n=N_1+1}^{\infty}(s_n-s)r^n\right|+(1-r)|s|\\
\leqslant&(1-r)\sum_{n=1}^{N_1}|s_n-s|r^n+(1-r)\varepsilon\sum_{n=N_1+1}^{\infty}r^n+(1-r)|s|\\
\leqslant&(1-r)\sum_{n=1}^{N_1}2Mr^n+(1-r)\varepsilon\sum_{n=0}^{\infty}r^n+(1-r)|s|\\
\leqslant&2MN_1(1-r)+\varepsilon+(1-r)|s|
\end{aligned}
$$
\par
由$1-r\to 0, r\to 1^-$,对上述$\varepsilon$,存在$\delta>0$,使得任意$1-\delta<r<1$,有:
$$
\left|(1-r)\sum_{n=1}^{\infty}s_nr^n-s\right|\lesssim \varepsilon
$$
\par
结论成立.
\par
此处用了调和分析符号$\lesssim$.
\par
$f(x)\lesssim g(x)$即存在$C>0$,使得$f(x)\leqslant Cg(x)$,对充分大的$x$成立.
\par
(2)
\par
考虑$c_n=(-1)^n$,$s_n=\sum_{i=1}^{n}c_n$.
\par
其$Abel$和为
$$
A_r=\sum_{n=1}^{\infty}(-1)^nr^n=\sum_{n=1}^{\infty}(-r)^n=\dfrac{-r}{1+r}
$$
\par
$$
\lim_{r \to 1^-}A_r=-\dfrac{1}{2}
$$
\par
但是显然$s_n$不收敛,即$\sum c_n$不收敛.
\par
(3)
\par
证明:
\par
设$\sigma_n=\dfrac{s_1+s_2+\cdots+s_n}{n}$.
利用$Abel$求和公式:记$s_n=\sum_{i=1}^{n}c_i$
$$
\begin{aligned}
&\sum_{n=1}^{N}c_n r^n\\
=&r^{N+1}s_N+(1-r)\sum_{n=1}^{N}s_nr^n\\
=&r^{N+1}s_N+(1-r)\left(r^NN\sigma_N+(1-r)\sum_{n=1}^{N-1}nr^n\sigma_n\right)\\
=&r^{N+1}s_N+r^NN\sigma_N-r^{N+1}N\sigma_N+(1-r)^2\sum_{n=1}^{N-1}n\sigma_nr^n\\
=&r^{N+1}(N\sigma_N-(N-1)\sigma_{N-1})+r^NN\sigma_N-r^{N+1}N\sigma_N+(1-r)^2\sum_{n=1}^{N-1}n\sigma_nr^n\\
=&r^{N}N\sigma_N-r^{N+1}(N-1)\sigma_{N-1}+(1-r)^2\sum_{n=1}^{N-1}n\sigma_nr^n
\end{aligned}
$$
\par
由于
$$
\lim_{n \to \infty}\sigma_n=\sigma
$$
\par
对任意$\varepsilon>0$,存在$N_2\in \mathbb{N}$,使得任意$n>N_2$,有
$$
|\sigma_n-\sigma|<\varepsilon
$$
\par
存在$L>0$,使得对于任意$n\in \mathbb{N}$有,
$$
|\sigma_n|\leqslant L, |\sigma|\leqslant L
$$
$r^{N}N\sigma_N$和$r^{N+1}(N-1)\sigma_{N-1}$以及$(1-r)^2\sum_{n=1}^{N-1}n\sigma_nr^n$
的极限都存在,这是因为根值判别法.
\par
令$N\to \infty$,有:
$$
\sum_{n=1}^{\infty}c_nr^n=(1-r)^2\sum_{n=1}^{\infty}n\sigma_nr^n
$$
\par
接下来只需要证明:
$$
(1-r)^2\sum_{n=1}^{\infty}n\sigma_nr^n=\sigma, r \to 1^-
$$
\par
我们继续作差:
$$
\begin{aligned}
&\left|(1-r)^2\sum_{n=1}^{\infty}n\sigma_nr^n-\sigma\right|\\
=&\left|(1-r)^2\sum_{n=1}^{\infty}n(\sigma_n-\sigma)r^n+\sigma(1-r)^2\sum_{n=1}^{\infty}nr^n-\sigma\right|\\
=&\left|(1-r)^2\sum_{n=1}^{\infty}n(\sigma_n-\sigma)r^n+\sigma (r-1)\right|\\
\leqslant&(1-r)^2\sum_{n=1}^{N_2}n|\sigma_n-\sigma|r^n
+(1-r)^2\sum_{n=N_2+1}^{\infty}n|\sigma_n-\sigma|r^n+(1-r)|\sigma|\\
\leqslant&(1-r)^2\cdot 2LN_1^{2}+\varepsilon+(1-r)|\sigma|
\end{aligned}
$$
\par
由于$(1-r)^2\to 0, (1-r)\to 0, r \to 1^-$,
\par
对上述$\varepsilon>0$,存在$\delta_0>0$,使得任意$1-\delta_0<r<1$有:
$$
\left|(1-r)^2\sum_{n=1}^{\infty}n\sigma_nr^n-\sigma\right|\lesssim \varepsilon
$$
\par
得证!
\par
(4)
\par
考虑$c_n=(-1)^nn$,则其$Abel$和为:
$$
A_r=\sum_{n=1}^{\infty}(-1)^nnr^n=\dfrac{r}{(1+r)^2}
$$
\par
$$
\lim_{r \to 1^-}A_r=\dfrac{1}{4}
$$
\par
而对于$Ces\grave{a}ro$可和级数$\sum c_n$,有$\dfrac{c_n}{n}\to 0, n \to \infty$
\par
这是因为
$$
\dfrac{s_1+s_2+\cdots+s_n}{n}\to \sigma, n\to  \infty
$$
\par
而
$$
\dfrac{c_n}{n}=\dfrac{s_n-s_{n-1}}{n}=\dfrac{n\sigma_n-(n-1)\sigma_{n-1}}{n}\to \sigma-\sigma=0,n\to \infty
$$
\par
但是$\dfrac{c_n}{n}=(-1)^n$不收敛,那么我们完成了反例.
\end{solution}


\begin{problem}
Exercise 14: 小$o$ Tauber型定理.
\par
(1)如果级数$\sum c_n$满足$Ces\grave{a}ro$可和收敛到$\sigma$,
且$c_n=o\left(\dfrac{1}{n}\right)$,证明:
$$
\sum_{n=1}^{\infty}c_n=\sigma
$$
\par
(2)如果级数$\sum c_n$满足$Abel$可和收敛到$s$,
且$c_n=o\left(\dfrac{1}{n}\right)$,证明:
$$
\sum_{n=1}^{\infty}c_n=s
$$
    
\end{problem}

\begin{solution}
\par
(1)证明:
\par
$$
s_n-\sigma_n=\dfrac{(n-1)c_n+(n-2)c_{n-1}+\cdots+c_2}{n}
$$
\par
又$c_n=o\left(\dfrac{1}{n}\right)$,则$(n-1)c_n=\dfrac{n-1}{n}\cdot nc_n\to 0, n\to \infty$
\par
则根据复版本Cauchy命题,
$$
s_n-\sigma_n\to 0, n \to \infty
$$
\par
那么
$s_n=s_n-\sigma_n+\sigma_n\to 0+\sigma=\sigma, n \to \infty$
\par
即
$$
\sum_{n=1}^{\infty}c_n=\sigma
$$
\par
(2)证明:
\par
根据$c_n=o\left(\dfrac{1}{n}\right)$,即$nc_n\to 0,n\to \infty$.
\par
对任意$\varepsilon>0$,存在$N_0\in \mathbb{N}$,使得任意$n>N_0$,有$|c_n|<\dfrac{\varepsilon}{n}$
\par
当$N>N_0$时,
$$
\begin{aligned}
&\left|\sum_{n=1}^{N}c_n-s\right|\\
=&\left|\sum_{n=1}^{N}c_n-\sum_{n=1}^{N}c_nr^n-
\sum_{n=N+1}^{\infty}c_nr^n+\sum_{n=1}^{\infty}c_nr^n-s\right|\\
=&\left|\sum_{n=1}^{N}c_n-\sum_{n=1}^{N}c_nr^n\right|
+\left|\sum_{n=N+1}^{\infty}c_nr^n\right|
+\left|\sum_{n=1}^{\infty}c_nr^n-s\right|\\
=&\left|\sum_{n=1}^{N}c_n(1-r^n)\right|+
\dfrac{1}{N}\sum_{n=N+1}^{\infty}n|c_n|r^n+
+\left|\sum_{n=1}^{\infty}c_nr^n-s\right|
\\
\leqslant&\left|\sum_{n=1}^{N}c_n(1-r)(1+r+\cdots+r^{n-1})\right|
+\dfrac{\varepsilon}{N}\sum_{n=N+1}^{\infty}r^n
+\left|\sum_{n=1}^{\infty}c_nr^n-s\right|\\
\leqslant &\sum_{n=1}^{N}|c_n|(1-r)(1+r+\cdots+r^{n-1})+
\dfrac{\varepsilon}{N}\dfrac{1}{1-r}+
\left|\sum_{n=1}^{\infty}c_nr^n-s\right|
\\
\leqslant &\sum_{n=1}^{N}|nc_n|(1-r)
+\dfrac{\varepsilon}{N}\dfrac{1}{1-r}+
\left|\sum_{n=1}^{\infty}c_nr^n-s\right|
\end{aligned}
$$
\par
令$r=1-\dfrac{1}{N}$,继续有:
$$
\begin{aligned}
    &\left|\sum_{n=1}^{N}c_n-\sum_{n=1}^{N}c_nr^n\right|\\
    \leqslant &\dfrac{\sum_{n=1}^{N}|nc_n|}{N}\to 0, N \to \infty
    \end{aligned}
$$
\par
这是因为复版本Cauchy命题和$nc_n\to 0, |nc_n|\to 0, n\to \infty$
\par
还有,
$$
\dfrac{\varepsilon}{N}\dfrac{1}{1-r}=\varepsilon
$$
\par
由于$$
\lim_{r \to 1^-}\sum_{n=1}^{\infty}c_nr^n= s
$$
,
$$
\lim_{N \to \infty}\sum_{n=1}^{\infty}c_n\left(1-\dfrac{1}{N}\right)^n= s
$$
\par
对上述$\varepsilon>0$,存在$N_1 \in \mathbb{N}$,使得任意$N>N_1$,有:
$$
\left|\sum_{n=1}^{N}c_n-s\right|\lesssim \varepsilon
$$
\par
那么有
$$
\sum_{n=1}^{\infty}c_n=\sum_{n=1}^{\infty}c_nr^n=s
$$
\end{solution}

\begin{note}
    \par
    注意此题和$Abel$定理的联系.
    \par
    此题条件可以继续减弱,得到大$O$ Tauber型定理.
\end{note}


\begin{problem}
    Exercise 16: Weierstrass第一定理
    \par
    证明:
    \par
    若$f$是闭区间$[a,b]$上的连续函数, 则对任意$\varepsilon>0$,
    存在多项式$P(x)$满足:
    $$
 \sup_{x\in[a,b]}|f(x)-P(x)|<\varepsilon
    $$
\end{problem}

\begin{solution}
\par
固定正数$l$,延拓$f(x)$到$[a-l,b+l]$,使得$f(a-l)=f(b+l)$,
再延拓为$\mathbb{R}$上的周期为$2l+b-a$的连续函数.
\par
改造之前的证明,我们知道其相当于圆周$\mathbb{R}/(2l+b-a)\mathbb{Z}$上的连续函数,
存在三角多项式$Q(x)$,使得
$$
\sup_{x\in[a-l,b+l]}|f(x)-Q(x)|<\dfrac{\varepsilon}{2}
$$
\par
又$\mathrm{e}^z$在紧集上被多项式一致逼近,则存在多项式$P(x)$,使得
$$
\sup_{x\in[a-l,b+l]}|Q(x)-P(x)|<\dfrac{\varepsilon}{2}
$$
\par
则
$$
\begin{aligned}
&\sup_{x\in[a-l,b+l]}|f(x)-P(x)|\\
\leqslant&\sup_{x\in[a-l,b+l]}\left(|f(x)-Q(x)|+|Q(x)-P(x)|\right)\\
\leqslant&\sup_{x\in[a-l,b+l]}|f(x)-Q(x)|+\sup_{x\in[a-l,b+l]}|Q(x)-P(x)|\\
<&\dfrac{\varepsilon}{2}+\dfrac{\varepsilon}{2}\\
=&\varepsilon
\end{aligned}
$$
\par
那么我们证明了:
$$
\begin{aligned}
\sup_{x\in[a,b]}|f(x)-P(x)|<\varepsilon
\end{aligned}
$$
\end{solution}

\begin{problem}
Exercise 17:
\par
我们开始研究$Abel$均值和$Ces\grave{a}ro$均值在不连续点的行为.
\par
我们称可积函数$f$在$\theta$处有跳跃间断点,即
$$
\lim_{h\to 0, h>0}f(\theta+h)=f(\theta^+), \quad \lim_{h \to 0,h>0}f(\theta-h)=f(\theta^-)
$$
\par
极限都存在且不相等.
\par
(1)证明:
\par
若可积函数$f$有一个在$\theta$处的跳跃间断点,则
$$
\lim_{r\to 1^-}A_r(f)(\theta)=\dfrac{f(\theta^+)+f(\theta^-)}{2}, 0\leqslant r <1
$$
\par
(2)证明:
\par
若可积函数$f$有一个在$\theta$处的跳跃间断点,则
$$
\lim_{n\to \infty}\sigma_n(f)(\theta)=\dfrac{f(\theta^+)+f(\theta^-)}{2}
$$
\par
提示:考虑Good kernel的性质.
\end{problem}

\begin{solution}
\par
(1)证明:
\par
我们考虑Good kernel的性质:
$$
\dfrac{1}{2\pi}\int_{0}^{\pi}P_r(\theta)\mathrm{d}\theta=\dfrac{1}{2}, 
\quad\dfrac{1}{2\pi}\int_{-\pi}^{0}P_r(\theta)\mathrm{d}\theta=\dfrac{1}{2}
$$
\par
这是因为$Poisson$核是Good kernel,并且是偶函数.
\par
可积函数必定有界, 设
$$
\sup_{x\in[-\pi,\pi]}|f(x)|\leqslant B
$$
\par
对任意$\varepsilon>0$, 存在$\delta>0$,使得任意$-\delta<x<0$,有
$|f(x)-f(\theta^-)|<\varepsilon$;
\par
对任意$0<x<\delta$,有$|f(x)-f(\theta^+)|<\varepsilon$
\par
$$
\begin{aligned}
&\left|A_r(f)(\theta)-\dfrac{f(\theta^+)+f(\theta^-)}{2}\right|\\
\leqslant&\left|P_r*f(\theta)-\dfrac{f(\theta^+)+f(\theta^-)}{2}\right|\\
=&\left|
\dfrac{1}{2\pi}\int_{-\pi}^{\pi}f(\theta-t)P_r(t)\mathrm{d}t
-\dfrac{1}{2\pi}\int_{0}^{\pi}f(\theta^+)P_r(t)\mathrm{d}t
-\dfrac{1}{2\pi}\int_{0}^{\pi}f(\theta^-)P_r(t)\mathrm{d}t
\right|\\
=&\left|
\dfrac{1}{2\pi}\int_{0}^{\pi}(f(\theta-t)-f(\theta ^-))P_r(t)\mathrm{d}t
-\dfrac{1}{2\pi}\int_{0}^{\pi}f(\theta^+)P_r(t)\mathrm{d}t
+\dfrac{1}{2\pi}\int_{-\pi}^{0}f(\theta-t)P_r(t)\mathrm{d}t\right|\\
=&\left|\dfrac{1}{2\pi}\int_{0}^{\pi}(f(\theta-t)-f(\theta ^-))P_r(t)\mathrm{d}t
+\dfrac{1}{2\pi}\int_{0}^{\pi}(f(\theta+t)-f(\theta ^+))P_r(t)\mathrm{d}t\right|\\
\leqslant&\dfrac{1}{2\pi}\int_{0}^{\delta}|f(\theta-t)-f(\theta^-)|P_r(t)\mathrm{d}t
+\dfrac{1}{2\pi}\int_{0}^{\delta}|f(\theta+t)-f(\theta^+)|P_r(t)\mathrm{d}t\\
+&\dfrac{1}{2\pi}\int_{\delta}^{\pi}|f(\theta+t)-f(\theta^+)|P_r(t)\mathrm{d}t
+\dfrac{1}{2\pi}\int_{\delta}^{\pi}|f(\theta+t)-f(\theta^+)|P_r(t)\mathrm{d}t\\
\leqslant&2\cdot\dfrac{1}{2\pi}\int_{0}^{\pi}\varepsilon P_r(t)\mathrm{d}t
+2B\cdot \dfrac{1}{2\pi}\int_{\delta \leqslant |t| \leqslant \pi}|P_r(t)|\mathrm{d}t\\
=&\varepsilon+2B\cdot \dfrac{1}{2\pi}\int_{\delta \leqslant |t| \leqslant \pi}|P_r(t)|\mathrm{d}t
\end{aligned}
$$
\par
又$$
\dfrac{1}{2\pi}\int_{\delta \leqslant |t| \leqslant \pi}|P_r(t)|\mathrm{d}t \to 0, r \to 1^- 
$$
\par
对上述$\varepsilon>0$,存在$\delta_0>0$,使得任意$1-\delta_0<r<1$,有
$$
\left|A_r(f)(\theta)-\dfrac{f(\theta^+)+f(\theta^-)}{2}\right|\lesssim \varepsilon
$$
\par
命题成立.
\par
\quad
\par
(2)证明:
\par
我们考虑Good kernel的性质:
$$
\dfrac{1}{2\pi}\int_{0}^{\pi}F_N(\theta)\mathrm{d}\theta=\dfrac{1}{2}, 
\quad\dfrac{1}{2\pi}\int_{-\pi}^{0}F_N(\theta)\mathrm{d}\theta=\dfrac{1}{2}
$$
\par
这是因为$Fej\acute{e}r$核是Good kernel,并且是偶函数.
\par
可积函数必定有界, 设
$$
\sup_{x\in[-\pi,\pi]}|f(x)|\leqslant B
$$
\par
对任意$\varepsilon>0$, 存在$\delta>0$,使得任意$-\delta<x<0$,有
$|f(x)-f(\theta^-)|<\varepsilon$;
\par
对任意$0<x<\delta$,有$|f(x)-f(\theta^+)|<\varepsilon$
\par
$$
\begin{aligned}
&\left|\sigma_N(f)(\theta)-\dfrac{f(\theta^+)+f(\theta^-)}{2}\right|\\
\leqslant&\left|F_N*f(\theta)-\dfrac{f(\theta^+)+f(\theta^-)}{2}\right|\\
=&\left|
\dfrac{1}{2\pi}\int_{-\pi}^{\pi}f(\theta-t)F_N(t)\mathrm{d}t
-\dfrac{1}{2\pi}\int_{0}^{\pi}f(\theta^+)F_N(t)\mathrm{d}t
-\dfrac{1}{2\pi}\int_{0}^{\pi}f(\theta^-)F_N(t)\mathrm{d}t
\right|\\
=&\left|
\dfrac{1}{2\pi}\int_{0}^{\pi}(f(\theta-t)-f(\theta ^-))F_N(t)\mathrm{d}t
-\dfrac{1}{2\pi}\int_{0}^{\pi}f(\theta^+)F_N(t)\mathrm{d}t
+\dfrac{1}{2\pi}\int_{-\pi}^{0}f(\theta-t)F_N(t)\mathrm{d}t\right|\\
=&\left|\dfrac{1}{2\pi}\int_{0}^{\pi}(f(\theta-t)-f(\theta ^-))F_N(t)\mathrm{d}t
+\dfrac{1}{2\pi}\int_{0}^{\pi}(f(\theta+t)-f(\theta ^+))F_N(t)\mathrm{d}t\right|\\
\leqslant&\dfrac{1}{2\pi}\int_{0}^{\delta}|f(\theta-t)-f(\theta^-)|F_N(t)\mathrm{d}t
+\dfrac{1}{2\pi}\int_{0}^{\delta}|f(\theta+t)-f(\theta^+)|F_N(t)\mathrm{d}t\\
+&\dfrac{1}{2\pi}\int_{\delta}^{\pi}|f(\theta+t)-f(\theta^+)|F_N(t)\mathrm{d}t
+\dfrac{1}{2\pi}\int_{\delta}^{\pi}|f(\theta+t)-f(\theta^+)|F_N(t)\mathrm{d}t\\
\leqslant&2\cdot\dfrac{1}{2\pi}\int_{0}^{\pi}\varepsilon F_N(t)\mathrm{d}t
+2B\cdot \dfrac{1}{2\pi}\int_{\delta \leqslant |t| \leqslant \pi}|F_N(t)|\mathrm{d}t\\
=&\varepsilon+2B\cdot \dfrac{1}{2\pi}\int_{\delta \leqslant |t| \leqslant \pi}|F_N(t)|\mathrm{d}t
\end{aligned}
$$
\par
又$$
\dfrac{1}{2\pi}\int_{\delta \leqslant |t| \leqslant \pi}|F_N(t)|\mathrm{d}t \to 0, N \to \infty 
$$
\par
对上述$\varepsilon>0$,存在$N_0>0$,使得任意$N>N_0$,有
$$
\left|\sigma_N(f)(\theta)-\dfrac{f(\theta^+)+f(\theta^-)}{2}\right|\lesssim \varepsilon
$$
\par
命题成立.
\end{solution}
\begin{problem}
Exercise 18: 
\par
$P_r(\theta)$表示$Poisson$核,证明:函数
$$
u(r,\theta)=\dfrac{\partial P_r}{\partial \theta}
$$
\par
定义在$0\leqslant r <1$和$\theta \in \mathbb{R}$上,满足
\par
(1)$$
\Delta u =0
$$
\par
在圆盘上成立;
\par
(2)$$
\lim_{r \to 1^-}u(r,\theta)=0
$$
\par
对所有$\theta \in \mathbb{R}$成立.
\end{problem}

\begin{solution}
    \par
    (1)
    \par
$$
\begin{aligned}
 u(r,\theta)=
 &\dfrac{\partial}{\partial \theta}\sum_{n=-\infty}^{\infty}r^{|n|}\mathrm{e}^{in\theta}\\
 =&\sum_{n=-\infty}^{\infty}inr^{|n|}\mathrm{e}^{in\theta}\\
\end{aligned}
 $$
 \par
 不断计算有:
 $$
 \begin{aligned}
 &\dfrac{\partial u}{\partial \theta}=\sum_{n=-\infty}^{\infty}(-n^2)r^{|n|}\mathrm{e}^{in\theta}\\
 &\dfrac{\partial^2 u}{\partial \theta^2}=\sum_{n=-\infty}^{\infty}(-in^3)r^{|n|}\mathrm{e}^{in\theta}\\
 &\dfrac{\partial u}{\partial r}=\sum_{n=-\infty}^{\infty}in|n|r^{|n|-1}\mathrm{e}^{in\theta}\\
 &\dfrac{\partial^2 u}{\partial r^2}=\sum_{n=-\infty}^{\infty}in(n^2-|n|)r^{|n|-2}\mathrm{e}^{in\theta}
\end{aligned}
 $$
 \par
 于是,
 $$
\dfrac{\partial^2 u}{\partial r^2}+\dfrac{1}{r}\dfrac{\partial u}{\partial r}+\dfrac{1}{r^2}\dfrac{\partial^2 u}{\partial \theta^2}=0
 $$
 \par
 满足$$
\Delta u=0
 $$
 \par
 在圆盘中成立.
 \par
 (2)我们有:
 $$
u(r,\theta)=
\dfrac{\partial}{\partial \theta}\left(\dfrac{1-r^2}{1-2r\cos \theta+r^2}\right)
=\dfrac{2r(r^2-1)\sin \theta}{(1-2r\cos \theta+r^2)^2}
 $$
\par
当$\theta=2k\pi$时,$u(r,\theta)=\dfrac{2r(r-1)(r+1)\cdot 0}{(r-1)^4}=0$
\par
$$
\lim_{r \to 1^-}u(r,\theta)=0
$$
\par
当$\theta \neq 2k\pi $时,$u(r,\theta)=\dfrac{2r(r^2-1)\sin \theta}{(1-2r\cos \theta +r^2)^2}$
$$
\lim_{r \to 1^-}u(r,\theta)=\dfrac{0}{2-2\cos \theta}=0.
$$
\par
命题得证.
\end{solution}


\begin{problem}
Problem 2:
\par
$D_N$是$N$阶Dirichlet核,
$$
D_N(\theta)=\sum_{k=-N}^{N}\mathrm{e}^{ik\theta}=
\dfrac{\sin((N+\frac{1}{2})\theta)}{\sin (\theta/2)}
$$
\par
定义:
$$
L_N=\dfrac{1}{2\pi}\int_{-\pi}^{\pi}|D_N(\theta)|\mathrm{d}\theta
$$
\par
(1)
证明:存在$c>0$,使得:
$$
L_N\geqslant \dfrac{4}{\pi^2}\log N+O(1)
$$
\par
(2)证明:对每一个$n\geqslant 1$, 存在一个连续函数$f_n$,
使得$|f_n|\leqslant 1$和$|S_n(f_n)(0)|\geqslant c\log n$.
\end{problem}

\begin{solution}
\par
(1)证明:首先熟知不等式$\sin x\leqslant x, x\geqslant 0 $,当且仅当$x=0$时取等.
\par
$$
\begin{aligned}
L_N
=&\dfrac{1}{2\pi}\int_{-\pi}^{\pi}|D_N(\theta)|\mathrm{d}\theta\\
=&\dfrac{1}{2\pi}\int_{-\pi}^{\pi}\dfrac{|\sin((N+\frac{1}{2})\theta)|}
{|\sin (\theta /2)|}\mathrm{d}\theta\\
=&\dfrac{1}{\pi}\int_{0}^{\pi}\dfrac{|\sin((N+\frac{1}{2})\theta)|}
{\sin (\theta /2)}\mathrm{d}\theta\\
\geqslant& \dfrac{2}{\pi}\int_{0}^{\pi}\dfrac{|\sin ((N+\frac{1}{2})\theta)|}{\theta}
\mathrm{d}\theta\\
=&\dfrac{2}{\pi}\int_{0}^{(N+\frac{1}{2})\pi}\dfrac{|\sin t|}{t}\mathrm{d}t\\
=& \dfrac{2}{\pi}\int_{0}^{N\pi}\dfrac{|\sin t|}{t}\mathrm{d}t+O(1)\\
=&\dfrac{2}{\pi}\sum_{k=1}^{N}\int_{(k-1)\pi}^{k\pi}\dfrac{|\sin t|}{t}\mathrm{d}t+O(1)\\
\geqslant & \dfrac{2}{\pi}\sum_{k=1}^{N}\int_{(k-1)\pi}^{k\pi}\dfrac{|\sin t|}{k\pi}\mathrm{d}t+O(1)\\
\geqslant & \dfrac{4}{\pi^2}\sum_{k=1}^{N}\dfrac{1}{k}+O(1)\\
\geqslant &\dfrac{4}{\pi^2}\int_{1}^{N}\dfrac{1}{x}\mathrm{d}x+O(1)\\
=&\dfrac{4}{\pi^2}\log N+O(1)
\end{aligned}
$$
\par
其中,
$$
\dfrac{2}{\pi}\int_{N\pi}^{(N+\frac{1}{2})\pi}\dfrac{|\sin t|}{t}\mathrm{d}t\leqslant
\dfrac{2}{\pi}\int_{N\pi}^{(N+\frac{1}{2})\pi}\frac{1}{t}\mathrm{d}t=\dfrac{2}{\pi}\log\left(1+\frac{1}{2N}\right)\leqslant\dfrac{2\log 2}{\pi} 
$$
\par
则
$$
\dfrac{2}{\pi}\int_{N\pi}^{(N+\frac{1}{2})\pi}\dfrac{|\sin t|}{t}\mathrm{d}t=O(1)
$$
\par
命题得证.
\par
(2)证明:设$$
g_n(\theta)=
\begin{cases}
1, &D_n(\theta)>0;\\
-1, &D_n(\theta)<0.
\end{cases}
$$
\par
可见不连续点为$D_n(\theta)$改变正负的点.
\par
根据
$$
D_n(\theta)=\dfrac{\sin((N+\frac{1}{2})\theta)}{\sin (\theta/2)}
$$
\par
的正负变化,可以知道$g_n(\theta)$不连续点只有有限多个,根据Riemann-Lebesgue引理:
\par
 $g_n(\theta)$在$[-\pi,\pi]$上Riemann可积.
\par
又$|g_n(\theta)|\leqslant 1$,
\
\par
根据可积函数可以被连续函数$L^1$逼近的引理:
\par
对任意固定的$n \geqslant 1$,存在连续函数列$\{f_k^{(n)}\}$,使得$|f_k^{(n)}|\leqslant 1$,且
\par
$$
\int_{-\pi}^{\pi}|g_n(\theta)-f_k^{(n)}(\theta)|\mathrm{d}\theta \to 0, k \to \infty
$$
\par
对于$\varepsilon=\dfrac{4}{\pi \sup{D_n(\theta)}}$,存在$k_n$使得:
$$
\dfrac{1}{2\pi}\int_{-\pi}^{\pi}|g_n(\theta)-f_{k_n}^{(n)}(\theta)|\mathrm{d}\theta
 <\dfrac{2}{\pi^2\sup{D_n(\theta)}}
$$
\par
根据上一问中推导过程,其中$O(1)$部分为正数,还有$D_n(\theta)$是偶函数:
$$
\begin{aligned}
S_n(g_n(0))=&\dfrac{1}{2\pi}\int_{-\pi}^{\pi}g_n(\theta)*D_n(-\theta)\mathrm{d}\theta\\
=&\dfrac{1}{2\pi}\int_{-\pi}^{\pi}g_n(\theta)*D_n(\theta)\mathrm{d}\theta\\
=&\dfrac{1}{2\pi}\int_{-\pi}^{\pi}|D_n(\theta)|\mathrm{d}\theta\\
\geqslant& \dfrac{4}{\pi^2}\sum_{i=1}^{n}\dfrac{1}{i}
\end{aligned}
$$
\par
我们完成最后一步:
$$
\begin{aligned}
&S_n\left(f_{k_n}^{(n)}(0)\right)\\
\geqslant &S_n(g_n(0))-\left|S_n(g_n(0))-S_n(f_{k_n}^{(n)}(0))\right|\\
=&S_n(g_n(0))-\left|\dfrac{1}{2\pi}\int_{-\pi}^{\pi}
(g_n(\theta)-f_{k_n}^{(n)}(\theta))D_n(\theta)\mathrm{d}\theta\right|\\
\geqslant &\dfrac{4}{\pi^2}\sum_{i=1}^{n}\dfrac{1}{i}-\dfrac{2}{\pi^2}\\
\geqslant &\dfrac{2}{\pi^2}\sum_{i=1}^{n}\dfrac{1}{i}\\
\geqslant & \dfrac{2}{\pi^2}\log n
\end{aligned}
$$
\par
我们记$$
f_n=f_{k_n}^{(n)}, n\geqslant 1, c=\dfrac{2}{\pi^2}
$$
\par
于是我们完成了命题.
\end{solution}




















\end{document}