\documentclass[12pt, a4paper, oneside]{ctexart}
\usepackage{amsmath, amsthm, amssymb, bm, color, framed, graphicx, hyperref, mathrsfs}

\title{\textbf{General Topology Chapter 1}}
\author{张浩然}
\date{\today}
\linespread{1.5}
\definecolor{shadecolor}{RGB}{241, 241, 255}
\newcounter{problemname}
\newenvironment{problem}{\begin{shaded}\stepcounter{problemname}\par\noindent\textbf{题目\arabic{problemname}. }}{\end{shaded}\par}
\newenvironment{solution}{\par\noindent\textbf{解答. }}{\par}
\newenvironment{note}{\par\noindent\textbf{题目\arabic{problemname}的注记. }}{\par}

\begin{document}

\maketitle



\begin{problem}
Exercise 1:
\par
设 $f : [0,1] \rightarrow \mathbb{R}$是连续函数,且$f(0) = f(1) = 0$.
\par
 考虑由函数$f$在$[0,1]$之间的图像、$x$轴上从$x=0$到$x=1$的线段组成的简单闭曲线$C$ .
 \par
 证明: 总可以在这条平面简单闭曲线$C$上找到四个点,使得它们为一个正方形的顶点.
\par
 提示: 考虑 $g(x) = f(x) -f(x +f(x)), h(x)=x+f(x)$. 
\end{problem}


\begin{solution}
\par
由于$C$是简单闭曲线,$f(x)$在$(0,1)$上没有零点且连续,则恒正或者恒负.
\par
不妨设$f(x)$恒正,否则考虑$-f(x)$即可.
根据$f(x)$在$[0,1]$连续,存在$x_0\in (0,1)$,使得$f(x_0)>0$为最大值.
\par
考虑$h(x)=x+f(x)$,在$[0,1]$上连续,则有$g(0)=0+f(0)=0, g(1)=1+f(1)=1$,根据介值定理,
存在$x_1 \in (0,1)$,使得$g(x_1)=x_0$, 即$x_1+f(x_1)=x_0$.
\par
此时,我们延拓$f(x)$,
$$
\begin{aligned}
  f(x)=
\begin{cases}
f(x),&x \in [0,1];\\
0,   &x\in (1 +\infty)
\end{cases}
\end{aligned}
$$
\par
延拓后,考虑$g(x)=f(x)-f(x+f(x))$,在$[0,1]$上连续.
$$
\begin{aligned}
&g(x_0)=f(x_0)-f(x_0+f(x_0))\geqslant 0\\
&g(x_1)=f(x_1)-f(x_1+f(x_1))=f(x_1)-f(x_0)\leqslant 0
\end{aligned}
$$
\par
根据介值定理,存在$x_2$可能介于$x_0$和$x_1$之间, 可能等于$x_0$或$x_1$,
使得$$
g(x_2)=0, f(x_2)=f(x_2+f(x_2))
$$
\par
简单闭曲线$C$上$(x_2, 0),(x_2+f(x_2),0),(x_2, f(x_2)), \left(x_2+f(x_2), f(x_2+f(x_2))\right)$
为一个正方形的顶点.
\end{solution}

\begin{problem}
Exercise 2:
\par
一个足球可以只用六边形组成吗?
\end{problem}


\begin{solution}
\par
答案是否定的.
\par
假设成立,有$n$个六边形,我们考虑Euler公式:
\par
几何体每个顶点由三个六边形贡献:$V=\dfrac{6n}{3}=2n$
\par
几何体每条棱由两个六边形贡献:$E=\dfrac{6n}{2}=3n$
\par
几何体每个面只由一个六边形贡献:$F=\dfrac{n}{1}=n$
\par
则$$
V-E+F=2n-3n+n=0\neq 2
$$
\par
显然不能组成足球.


\end{solution}


\begin{problem}
Exercise 3:
\par
Weierstrasss在1870年对Dirichlet原理的反例
\par
对任意函数$u\in \mathcal{A}=\{ C^1[-1,1] : u(-1)=0, u(1)=1\}$,
定义:
$$
F(u)=\int_{-1}^{1}|xu'(x)|^2\mathrm{d}x
$$
\par
(1) 证明: 对任意$n\in \mathbb{N}$, 函数
$$
u_n(x) := \left(\sin \dfrac{n\pi x}{2}\right)^2 \chi _{\left[0,\frac{1}{n}\right]}(x) +
\chi_{\left(\frac{1}{n},1\right]}(x)
$$
\par
是 $\mathcal{A}$ 的元素, 其中$\chi_{A}(x)$是集合$A$的示性函数.
\par
(2) 证明: $$
\lim_{n \to \infty}
F(u_n) = 0.
$$
\par
(3) 证明: 不存在函数$u\in \mathcal{A}$可以取得$F$的最小值.
\end{problem}

\begin{solution}
\par
(1)证明:
$$
\begin{aligned}
u_n(x)=
\begin{cases}
  \left(\sin \dfrac{n\pi x}{2}\right)^2, &x\in \left[0,\frac{1}{n}\right]\\
  1,         &x \in  \left(\frac{1}{n},1\right]\\
  0,          &x \in [-1,0)
\end{cases}
\end{aligned}
$$

$$
\begin{aligned}
u_n'(x)=
\begin{cases}
  n\pi \sin \dfrac{n\pi x}{2}\cos \dfrac{n\pi x}{2}, &x\in \left(0,\frac{1}{n}\right)\\
  0,         &x \in  \left(\frac{1}{n},1\right)\\
  0,          &x \in (-1,0)
\end{cases}
\end{aligned}
$$
\par
容易验证:
$$
u_n'(0)=0=\lim_{x\to 0}u_n'(x), 
u_n'\left(\frac{1}{n}\right)=0=\lim_{x\to \frac{1}{n}}u_n'(x), 
$$
\par
于是$u_n\in C^1[-1,1]$且$u(-1)=0, u(1)=1$.
\par
则$u_n(x)\in \mathcal{A}, \forall n \in \mathbb{N}$.



\par
(2)证明:
\par
$$
\begin{aligned}
F(u_n)&=\int_{-1}^{1}|xu_n'(x)|^2\mathrm{d}x\\
&=\int_{0}^{\frac{1}{n}}\left(n\pi x\sin \frac{n\pi x}{2}\cos \frac{n\pi x}{2}\right)^2\mathrm{d}x\\
&\leqslant \frac{1}{4}\int_{0}^{\frac{1}{n}}n^2\pi^2x^2\mathrm{d}x\\
&=\frac{\pi^2}{12n}
\end{aligned}
$$
\par
于是,$$
0\leqslant F(u_n)\leqslant \frac{\pi^2}{12n}\to 0, n \to \infty
$$
$$
\lim_{n \to \infty}F(u_n)=0.
$$







\par
(3)证明:
\par
我们用反证法,假设存在函数$u$满足条件$F(u)=0$,
则
$$
\int_{-1}^{1}|xu'(x)|^2\mathrm{d}x=0
$$
\par
根据$u(x)\in C^1[-1,1]$, 则$|xu'(x)|^2\in C^0[-1,1]$,
则
$$
|xu'(x)|^2=0, \forall x \in [-1,1]
$$
\par
则$u'(x)=0, \forall x \in [-1,1]$,
$u(x)$为常数,但与$u(-1)=0, u(1)=1$矛盾!
\par
于是不存在函数$u$取得$F$的下确界.


\end{solution}


\end{document}