\documentclass[12pt, a4paper, oneside]{ctexart}
\usepackage{amsmath, amsthm, amssymb, bm, color, framed, graphicx, hyperref, mathrsfs}

\title{\textbf{Fourier Analysis Appendix 2}}
\author{张浩然}
\date{\today}
\linespread{1.5}
\definecolor{shadecolor}{RGB}{241, 241, 255}
\newcounter{problemname}
\newenvironment{problem}{\begin{shaded}\stepcounter{problemname}\par\noindent\textbf{题目\arabic{problemname}. }}{\end{shaded}\par}
\newenvironment{solution}{\par\noindent\textbf{解答. }}{\par}
\newenvironment{note}{\par\noindent\textbf{题目\arabic{problemname}的注记. }}{\par}

\begin{document}

\maketitle



\begin{problem}
  经典的核函数问题
  \par
  已知函数$f$在$[0,1]$上黎曼可积,在$x=0$处连续,证明:
  $$
     \lim_{h \to 0^+}\int_0^1\dfrac{h}{h^2+x^2}f(x)\mathrm{d}x=\dfrac{\pi}{2}f(0)
  $$
\end{problem}



\begin{solution}
 \par
 首先,我们为了更好地估计,我们只要处理:
 $$
 \left|\int_{0}^{1}\dfrac{h}{h^2+x^2}f(x)\mathrm{d}x-\dfrac{\pi}{2}f(0) \right|
 $$

\par
为了让绝对值内对象有相同的形式,我们注意到:

$$
\int_{0}^{1}\dfrac{h}{h^2+x^2}\mathrm{d}x=\arctan\left(\dfrac{1}{h}\right)
$$

\par
我们通过Taylor展开有:
$$
\int_{0}^{1}\dfrac{h}{h^2+x^2}\mathrm{d}x=\arctan\left(\dfrac{1}{h}\right)=\dfrac{\pi}{2}+O(h)
$$
\par
当然,我们可以计算:
$$
\lim_{h \to 0^+}\dfrac{\arctan\left(\dfrac{1}{h}\right)-\dfrac{\pi}{2}}{h}=-1
$$
\par
这就证明了结论.

\par
接下来,我们做替换:
$$
\begin{aligned}
 &\left|\int_{0}^{1}\dfrac{h}{h^2+x^2}f(x)\mathrm{d}x-\dfrac{\pi}{2}f(0) \right|\\
 =&\left|\int_{0}^{1}\dfrac{h}{h^2+x^2}(f(x)-f(0))\mathrm{d}x+O(h)\right|\\
 \leqslant &\int_{0}^{1}\dfrac{h}{h^2+x^2}|f(x)-f(0)|\mathrm{d}x+|O(h)|\\
\end{aligned}
$$

\par
根据$f$在$x=0$处连续,
对任意$\varepsilon>0$,存在$\delta\in(0,1)$,使得$|f(x)-f(0)|<\varepsilon$.
\par

$$
\begin{aligned}
 &\left|\int_{0}^{1}\dfrac{h}{h^2+x^2}f(x)\mathrm{d}x-\dfrac{\pi}{2}f(0) \right|\\
 < & \varepsilon\int_{0}^{\delta}\dfrac{h}{h^2+x^2}\mathrm{d}x+
 \int_{\delta}^{1}\dfrac{h}{h^2+x^2}|f(x)-f(0)|\mathrm{d}x+|O(h)|\\
 \leqslant &\varepsilon\int_{0}^{\delta}\dfrac{h}{h^2+x^2}\mathrm{d}x+
 2\sup_{x \in[0,1]}|f(x)|\int_{\delta}^{1}\dfrac{h}{h^2+x^2}\mathrm{d}x+|O(h)|\\
 =&\varepsilon \arctan\left(\dfrac{\delta}{h}\right)+2\sup_{x\in[0,1]}|f(x)|\left(\arctan\left(\dfrac{1}{h}\right)
 -\arctan\left(\dfrac{\delta}{h}\right)\right)+|O(h)|
\end{aligned}
$$
\par
其中,由微分中值定理:
$$
\arctan\left(\dfrac{1}{h}\right)-\arctan\left(\dfrac{\delta}{h}\right)
=\dfrac{1}{1+\xi^2}, \xi \in\left(\dfrac{\delta}{h},\dfrac{1}{h}\right)
$$
\par
有极限:
$$
\lim_{h \to 0^+}\arctan\left(\dfrac{1}{h}\right)-\arctan\left(\dfrac{\delta}{h}\right)
=\lim_{\xi \to +\infty}\dfrac{1}{1+\xi^2}=0
$$
\par
并且:
$$
|O(h)|\leqslant C|h|, C>0
$$
\par
有极限:
$$
\lim_{h \to 0^+}|O(h)|=0
$$
\par
于是,任意$\varepsilon>0$,存在$\delta_o>0$,使得
$$
\begin{aligned}
&\left|\int_{0}^{1}\dfrac{h}{h^2+x^2}f(x)\mathrm{d}x-\dfrac{\pi}{2}f(0) \right|\\
 < & \varepsilon \cdot \dfrac{\pi}{2}+\varepsilon+\varepsilon\\
 =&\left(\dfrac{\pi}{2}+2\right)\varepsilon
\end{aligned}
 $$
\par
综上所述,$$
\lim_{h \to 0^+}\int_0^1\dfrac{h}{h^2+x^2}f(x)\mathrm{d}x=\dfrac{\pi}{2}f(0)
$$

\end{solution}




\par
\quad
\par
\begin{problem}
我们现在考虑定义在$[-\pi,\pi]$上的一族可积函数$\{f_n(x)\}$:
\par
若满足以下三条性质:
\par
(1)
对所有$n \geqslant 1$有:
$$
\dfrac{1}{2\pi}\int_{-\pi}^{\pi}f_n(x)\mathrm{d}x=1
$$
\par
(2)
存在$M>0$,使得对所有$n \geqslant 1$有:
$$
\int_{-\pi}^{\pi}|f_n(x)|\mathrm{d}x\leqslant M
$$
\par
(3)对任意$\delta>0$,有:
$$
\lim_{n \to \infty}\int_{\delta \leqslant |x| \leqslant \pi}|f_n(x)|\mathrm{d}x=0
$$
\par
我们称$\{f_n\}$为好核(Good Kernel),或者渐近单位元(Approximation to the identity)
\end{problem}
\par
\quad
\par
容易修改$\dfrac{h}{h^2+x^2}$的系数,并且用连续指标$h$替代离散指标$n$,相应积分区间也替换后,
那么我们会得到一个好核,和上述定义唯一差别为指标的不同.
\par
如果我们引入卷积:
$$
(g*f)(x)=\dfrac{1}{2\pi}\int_{-\pi}^{\pi}g(x-y)f(y)\mathrm{d}y
$$

\par
并且$g$在相应区间可积,在$x_0$处连续,$\{f_n\}$是一个好核,那么卷积满足:
$$
\lim_{n \to \infty}(g*f_n)(x_0)=
\lim_{n \to \infty}\dfrac{1}{2\pi}\int_{-\pi}^{\pi}g(x_0-y)f_n(y)\mathrm{d}y=g(x_0)
$$

\par
此定理证明不难,留作练习.
\par
进一步的,如果$f$正则性更好,若处处连续,那么
$$
\lim_{n \to \infty}(g*f_n)(x)=g(x), \forall x \in [-\pi, \pi]
$$
\par
并且该极限是一致的.


\par
\quad
\par
\begin{problem}
  Weierstrass 第二逼近定理
  \par
证明:圆周上的连续函数可以被三角多项式一致逼近.
\end{problem}
\begin{solution}
对于圆周上的连续函数$f$,$S_N$是其傅里叶级数的前$N$项,可以定义:
$$
\sigma_Nf(x)=\dfrac{S_0f(x)+\cdots+S_{N-1}f(x)}{N}=(f*F_N)(x)
$$
\par
其中$\{F_N\}$是Fej$\acute{e}$r核,是一个好核,那么根据上述定理,
$$
\lim_{N \to \infty}\sigma_Nf(x)=f(x)
$$
\par
此极限是关于$x$一致的,并且$\sigma_N(f)(x)$是一个三角多项式,于是我们证明了结论.
\end{solution}


\end{document}